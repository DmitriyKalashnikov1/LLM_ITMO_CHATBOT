\documentclass[a4paper,14pt]{extarticle}


\usepackage[left = 20mm, right = 20mm, top = 25mm, bottom = 25mm, headheight=30pt]{geometry}
\usepackage{amssymb}
\usepackage{cmap}
\usepackage{enumerate}
\usepackage{latexsym}
\usepackage{amsmath}
\usepackage{euscript}
\usepackage{graphics}
\usepackage[T1, T2A]{fontenc}
\usepackage[utf8]{inputenc}
\usepackage[russian]{babel}
\usepackage[usenames]{color}
\usepackage{colortbl}
\usepackage{pgf,tikz}
\usepackage{multicol}
\usepackage{float}
\restylefloat{table}
\usepackage{multirow}
\usepackage{caption}
\usepackage{listings}
\usepackage{hyperref}
\usepackage{fancyhdr} 
\usepackage{subcaption}
\pagestyle{fancy}
\fancyhead{}
\fancyfoot{}
\fancyhead[LE,RO]{Университет ИТМО}
\fancyhead[RE,LO]{Высшая школа цифровой культуры}
\fancyfoot[LE,RO]{\thepage}
\renewcommand{\headrulewidth}{2pt}


\sloppy



\usepackage{xcolor} % Required for specifying custom colours
\definecolor{grey}{rgb}{0.9,0.9,0.9} % Colour of the box surrounding the title
%\usepackage[sfdefault]{ClearSans} % Use the Clear Sans font (sans serif)
%\usepackage{XCharter} % Use the XCharter font (serif)
\definecolor{foo}{HTML}{EFF5F9}

\DeclareMathOperator*{\argmax}{arg\,max}
\DeclareMathOperator*{\argmin}{arg\,min}
\DeclareMathOperator*{\Argmin}{Arg\,min}
\DeclareMathOperator*{\Argmax}{Arg\,max}
\DeclareMathOperator*{\logloss}{logloss}
\newcommand{\eqdef}{\overset{\mathrm{def}}{=\joinrel=}}
\def\re{{\rm Re}}
\def\im{{\rm Im}}
\def\dim{\rm dim}
\def\Ext{\rm Ext}
\def\wt#1{{{\widetilde #1} }}
\def\wh#1{{{\,\widehat #1\,} }}
\def\graph{{\rm gr\,}}
\def\ran{{\rm ran\,}}
\def\dom{{\rm dom\,}}
\def\ker{{\rm ker\,}}
\def\supp{{\rm supp\,}}
\def\diag{{\rm diag\,}}

\newcommand\dN{{\mathbb{N}}}
\newcommand\dR{{\mathbb{R}}}
\newcommand\dC{{\mathbb{C}}}
\newcommand{\bO}{{\mathbb{O}}}
\newcommand{\bU}{{\mathbb{U}}}
\newcommand\dZ{{\mathbb{Z}}}

\newcommand\gotB{{\mathfrak{B}}}
\newcommand\gotD{{\mathfrak{D}}}
\newcommand\gotH{{\mathfrak{H}}}
\newcommand\gotK{{\mathfrak{K}}}
\newcommand\gotL{{\mathfrak{L}}}
\newcommand\gotM{{\mathfrak{M}}}
\newcommand\gotN{{\mathfrak{N}}}
\newcommand\gotR{{\mathfrak{R}}}
\newcommand\gotS{{\mathfrak{S}}}
\newcommand\gotT{{\mathfrak{T}}}
\newcommand\gott{{\mathfrak{t}}}
\newcommand\gotC{{\mathfrak{C}}}
\newcommand\gotZ{{\mathfrak{Z}}}
\newcommand{\RNumb}[1]{\uppercase\expandafter{\romannumeral #1\relax}}

\newcommand{\ga}{{\alpha}}
\newcommand{\gd}{{\delta}}
\newcommand{\gD}{{\Delta}}
\newcommand{\gga}{{\gamma}}
\newcommand{\gG}{{\Gamma}}
\newcommand{\gF}{{\Phi}}
\newcommand{\gf}{{\phi}}
\newcommand{\gk}{{\kappa}}
\newcommand{\gK}{{\Kappa}}
\newcommand{\gl}{{\lambda}}
\newcommand{\gL}{{\Lambda}}
\newcommand{\gO}{{\Omega}}
\newcommand{\go}{{\omega}}
\newcommand{\gs}{{\sigma}}
\newcommand\gS{{\Sigma}}
\newcommand{\gT}{{\Theta}}
\newcommand{\gY}{{\Upsilon}}


\newcommand\cA{{\mathcal{A}}}
\newcommand\cB{{\mathcal{B}}}
\newcommand\cC{{\mathcal{C}}}
\newcommand\cD{{\mathcal{D}}}
\newcommand\cH{{\mathcal{H}}}
\newcommand\cK{{\mathcal{K}}}
\newcommand\cM{{\mathcal{M}}}
\newcommand\cN{{\mathcal{N}}}
\newcommand\cO{{\mathcal{O}}}
\newcommand\cP{{\mathcal{P}}}
\newcommand\cT{{\mathcal{T}}}
\newcommand\cU{{\mathcal{U}}}
\newcommand\cZ{{\mathcal{Z}}}

\newcommand\Rg{{\rm Rg}}
\newcommand\res{{\rm res}}
\newcommand\pr{{\rm Pr}}

\DeclareMathOperator{\sign}{sign}

\newtheorem{theorem}{Теорема}[subsection]
\newtheorem{lemma}{Лемма}[subsection]
\newtheorem{corollary}[theorem]{Следствие}
\newtheorem{definition}{Определение}[subsection]
\newtheorem{example}{Пример}[subsection]
\newtheorem{remark}{Замечание}[subsection]

\newcommand{\ba}{\begin{array}}
\newcommand{\ea}{\end{array}}

\newcommand{\bea}{\begin{eqnarray}}
\newcommand{\eea}{\end{eqnarray}}

\newcommand{\bead}{\begin{eqnarray*}}
\newcommand{\eead}{\end{eqnarray*}}

\newcommand{\be}{\begin{equation}}
\newcommand{\ee}{\end{equation}}

\newcommand{\bed}{\begin{displaymath}}
\newcommand{\eed}{\end{displaymath}}

\newcommand{\bl}{\begin{lemma}}
\newcommand{\el}{\end{lemma}}

\newcommand{\bt}{\begin{theorem}}
\newcommand{\et}{\end{theorem}}

\newcommand{\bc}{\begin{corollary}}
\newcommand{\ec}{\end{corollary}}

\newcommand{\br}{\begin{remark}}
\newcommand{\er}{\end{remark}}

\newcommand{\bd}{\begin{definition}}
\newcommand{\ed}{\end{definition}}

\newcommand{\bspi}{\begin{split}}
\newcommand{\espi}{\end{split}}

\newcommand{\la}{\label}
\newcommand{\rpm}{\raisebox{.2ex}{$\scriptstyle\pm$}}


\newenvironment{proof}%
{\begin{sloppypar}\noindent{\bf Доказательство.}}%
{\hspace*{\fill}$\square$\end{sloppypar}}
\renewcommand{\Large}{\fontsize{16}{25pt}\selectfont}

\newcommand{\slim}{\,\mbox{\rm s-}\hspace{-2pt} \lim}
\newcommand{\wlim}{\,\mbox{\rm w-}\hspace{-2pt} \lim}
\newcommand{\olim}{\,\mbox{\rm o-}\hspace{-2pt} \lim}
\newcommand{\transpose}[1]{\ensuremath{#1^{\scriptscriptstyle t}}}




\graphicspath{ {img/} }

\usepackage{indentfirst} 
\setlength\parindent{1cm}

\usepackage{textcase} 
\usepackage{titlesec}

\usepackage{indentfirst} 
\setlength\parindent{1cm}
% НАСТРОЙКИ ЗАГОЛОВКОВ
\usepackage{textcase} 
\usepackage{titlesec}
\titleformat{\section}[block]{\sffamily\Large\bfseries\filcenter}{\thesection}{0.5em}{}
\titleformat{\subsection}[block]{\large\sffamily\bfseries}{\thesubsection}{0.5em}{}
\titlespacing{\subsection}{2cm}{1mm}{3mm}

\begin{document}

%----------------------------------------------------------------------------------------
%	TITLE PAGE
%----------------------------------------------------------------------------------------

\begin{titlepage} % Suppresses displaying the page number on the title page and the subsequent page counts as page 1
	
	%------------------------------------------------
	%	Grey title box
	%------------------------------------------------
	\begin{center}
	\includegraphics[width=0.4\textwidth]{dc.png}
    \end{center}
    
    	\vfill
    	
	\colorbox{foo}{
		\parbox[t]{0.93\textwidth}{ % Outer full width box
			\parbox[t]{0.91\textwidth}{ % Inner box for inner right text margin
				\raggedleft % Right align the text
				%\fontsize{50pt}{80pt}\selectfont % Title font size, the first argument is the font size and the second is the line spacing, adjust depending on title length
				\vspace{0.7cm} % Space between the start of the title and the top of the grey box
				
				\huge Ансамбли моделей\\
				
				\vspace{0.7cm} % Space between the end of the title and the bottom of the grey box
			}
		}
	}
	
	\vfill % Space between the title box and author information
	
	%------------------------------------------------
	%	Author name and information
	%------------------------------------------------

    

	
	\parbox[t]{0.93\textwidth}{ % Box to inset this section slightly
		\raggedleft % Right align the text
		%\large % Increase the font size
		{Высшая Школа Цифровой Культуры}\\[4pt] % Extra space after name
		Университет ИТМО\\[4pt] % Extra space before URL
		dc@itmo.ru\\[4pt]
		
		\hfill\rule{0.2\linewidth}{1pt}% Horizontal line, first argument width, second thickness
	}
	
\end{titlepage}

%----------------------------------------------------------------------------------------


\newpage
%
\pagestyle{empty}
\tableofcontents
%
\clearpage
\pagestyle{fancy}


\subsection{Бутстрэп}
Давайте рассмотрим еще один из методов генерации повторных псевдовыборок -- бутстрэп. Этот метод был предложен в 1979 году Брэдли Эфроном (Bradley Efron), и с тех пор заработал заслуженную популярность. Он, как и джекнайф, позволяет оценивать различные характеристики рассматриваемой статистики такие, как смещение и дисперсию, позволяет строить доверительные интервалы и так далее. В то же время, границы применимости этого метода намного шире, чем метода джекнайф, естественно, за счет большей вычислительной сложности.

\subsubsection{Общая идея оценок методом бутстрэп}
Итак, попробуем самостоятельно, из логических соображений прийти к методу бутстрэп. Пусть, как обычно, $\theta$ -- некоторый неизвестный параметр (или характеристика) распределения генеральной совокупности $\xi$, $X_1, X_2, ..., X_n$ -- выборка из генеральной совокупности $\xi$, а
$$
\widehat \theta = \widehat \theta_n(X_1, X_2, ..., X_n)
$$
-- некоторая (состоятельная) оценка $\theta$. На конкретной выборке (в результате эксперимента) мы можем получить лишь конкретное (одно) значение нашей оценки $\widehat \theta$, а значит никак не можем оценить ни среднее, ни разброс, ни, тем более, построить какой-либо доверительный интервал для $\theta$. Как же поступать в таком случае?

Сначала представим себе <<идеальную ситуацию>>. Пусть мы можем получить не одну выборку из генеральной совокупности, а, скажем, $B$ штук. Тогда у нас в руках $B$ наборов 
$$
X_1^{(j)}, X_2^{(j)}, ..., X_n^{(j)}, \quad j \in \{1, 2, ..., B\}
$$
независимых и одинаково (с $\xi$) распределенных случайных величин, на каждом из которых мы можем вычислить значение нашей статистики $\widehat \theta$
$$
\widehat \theta^{j} = \widehat \theta_n^{j}(X_1^{(j)}, X_2^{(j)}, ..., X_n^{(j)}), \quad j \in \{1, 2, ..., B\}.
$$
Совершенно ясно, что разумной оценкой как математического ожидания, так и дисперсии $\widehat \theta$ являются
$$
\widehat{\theta}_{(\bullet),B} = \frac{1}{B}\sum\limits_{j = 1}^B \widehat \theta^{j}
$$
и
$$
\widehat{\mathsf{Var}}_B = \frac{1}{B-1}\sum\limits_{j = 1}^B\left(\widehat \theta^j - \frac{1}{B}\sum\limits_{j = 1}^B \widehat \theta^{j}\right)^2 = \frac{1}{B-1}\sum\limits_{j = 1}^B\left(\widehat \theta^j - \widehat{\theta}_{(\bullet),B}\right)^2.
$$
Все сказанное ранее подтверждается и теоретически -- изложенное, по сути своей, -- это хорошо известный в статистике метод Монте-Карло. Все бы было хорошо, но перед нами маячит вот какая проблема: обычно мы не можем получать сколько угодно выборок из генеральной совокупности. Что же в таком случае делать? Как применить, вроде бы, совершенно логичные рассуждения, которые мы привели? Тут и возникает бутстрэп.

\subsubsection{Метод бутстрэп}
Метод бутстрэп основывается вот на какой идее. Пусть $X_1, X_2, ..., X_n$ -- выборка. Опираясь на нее, мы можем смоделировать распределение $\xi$ и в последствии получить необходимые выборки из смоделированного распределения. Такой подход позволит нам  применить аппарат, описанный ранее.

Как известно из статистики, истинное распределение $\xi$ по выборке $X_1, X_2, ..., X_n$ разумно приближается эмпирическим с функцией распределения
$$
F_n^\ast(x) = \frac{1}{n}\sum\limits_{i = 1}^n \mathsf I(X_i < x),
$$
где $\mathsf I(A)$ -- индикатор события $A$, аналитическое задание которого следующее:
$$
\mathsf I(A) = \begin{cases}
 1, & A \text{ произошло}\\
 0, & A \text{ не произошло}
\end{cases}.
$$
Имея же функцию распределения, мы легко можем моделировать выборки нужного нам объема. Но чему, по сути дела, это эквивалентно? Конечно, формированию выборок путем случайного выбора $n$ объектов с возвращением из исходной выборки.
\begin{definition}
Бутстрэп-выборками объема $n$ из выборки $X_1, X_2, ..., X_n$ называются наборы
$$
X_1^{(j)}, X_2^{(j)}, ..., X_n^{(j)},
$$	
где $X_i^{(j)} \in \{X_1, X_2, ..., X_n\}$.
\end{definition}
Понятно, что число таким образом полученных выборок чрезвычайно велико! Индекс $j$ может принимать значения из диапазона $\{1, 2, ..., n^n\}$ и, например, для выборки объема $12$ количество бутстрэп-выборок равно
$$
12^{12} = 8\ 916\ 100\ 448\ 256
$$
-- числу, даже названия разрядов которого знает далеко не каждый. Поэтому на практике, конечно, берут куда меньшее бутстрэп-выборок, чем существует на самом деле.

\begin{remark}
Так как выбор производится с возвращением, то в бутстрэп-выборках могут встречаться повторяющиеся элементы исходной выборки. Интересно, что с ростом $n$ доля уникальных элементов исходной выборки, встречающихся в бутстрэп-выборке, стремится к $\approx 0.632$.

Пусть мы <<набираем>> выборку $S_1, S_2, ..., S_n$. Для каждого элемента исходной выборки $X_1, X_2, ..., X_n$ вероятность, что $S_i \neq X_j$ равна
$$
\mathsf P \left(S_i \neq X_j \right) = 1 - \frac{1}{n}.
$$
Так как выбор производится независимо, то
$$
\mathsf P(S_1 \neq X_j, S_2 \neq X_j, ..., S_n \neq S_j) = \left( 1 - \frac{1}{n}\right)^n \xrightarrow[n \to + \infty]{} e^{-1} \approx 0.368.
$$
Значит, вероятность того, что элемент $X_j$ встретится в набираемой нами выборке равна
$$
1 - e^{-1} \approx 0.632,
$$
и это справедливо для любого элемента исходной выборки, то есть при всех $j \in \{1, 2, ..., n\}$.
\end{remark}

Итак, после генерации бутстрэп-выборок, алгоритм получения оценок математического ожидания и дисперсии оценки $\widehat \theta$ следующий. 
\begin{enumerate}
\item Пусть сгенерировано $B$ бутстрэп-выборок 
$$
X_1^{(j)}, X_2^{(j)}, ..., X_n^{(j)}, \quad j \in \{1, 2, ..., B\}.
$$
\item На каждой из них вычисляется значение $\widehat \theta$, 
$$
\widehat \theta^{j} = \widehat \theta_n^{j}(X_1^{(j)}, X_2^{(j)}, ..., X_n^{(j)}), \quad j \in \{1, 2, ..., B\}.
$$
\item В качестве оценок математического ожидания и дисперсии $\widehat \theta$ берутся
$$
\widehat{\theta}_{(\bullet),B} = \frac{1}{B}\sum\limits_{j = 1}^B \widehat \theta^{j}
$$
и
$$
\widehat{\mathsf{Var}}_B = \frac{1}{B-1}\sum\limits_{j = 1}^B\left(\widehat \theta^j - \frac{1}{B}\sum\limits_{j = 1}^B \widehat \theta^{j}\right)^2 = \frac{1}{B-1}\sum\limits_{j = 1}^B\left(\widehat \theta^j - \widehat{\theta}_{(\bullet),B}\right)^2,
$$
соответственно. 
\end{enumerate}

\begin{definition}
Введенные выше величины 	
$$
\widehat{\theta}_{(\bullet),B} = \frac{1}{B}\sum\limits_{j = 1}^B \widehat \theta^{j}
$$
и
$$
\widehat{\mathsf{Var}}_B = \frac{1}{B-1}\sum\limits_{j = 1}^B\left(\widehat \theta^j - \frac{1}{B}\sum\limits_{j = 1}^B \widehat \theta^{j}\right)^2 = \frac{1}{B-1}\sum\limits_{j = 1}^B\left(\widehat \theta^j - \widehat{\theta}_{(\bullet),B}\right)^2,
$$
называют бутстрэп-оценками математического ожидания и дисперсии оценки $\widehat  \theta$, соответсвенно.
\end{definition}


\begin{example}
Рассмотрим уже знакомый пример, когда генеральная совокупность имеет распределение $\mathsf U_{\theta, \theta + 1}$ с параметром $\theta = 2.5$. 
Выполним моделирование и найдем истинное значения дисперсии оценки, а также джекнайф и бутстрэп-оценки дисперсии $\widehat \theta = X_{(1)}$.
\begin{figure}[h!]
	\centering\includegraphics[width=1\linewidth]{jack_uniform_var.png}
	\caption{Зависимость оценок дисперсии $\widehat{\mathsf{Var}}_{jack}$ и $\widehat{\mathsf{Var}}_{B}$ от объема выборки $n$}
		\label{jack_uniform_var}
\end{figure}
Можно заметить, что при малых значениях $n$ результаты варьируются, однако зачастую джекнайф-оценка дисперсии меньше. И наоборот, начиная с некоторого $n$, бутстрэп-оценка дисперсии меньше или совпадает с джекнайф-оценкой дисперсии. 
Практика показывает, что метод джекнайф лучше показывает себя на малых выборках, а бутстрэп предпочтителен для больших объемов данных.
\end{example}	


\subsubsection{Построение доверительных интервалов}
Построение доверительных интервалов для $\theta$ опирается на ровно те же самые идеи, что озвучены ранее. Рассмотрим здесь лишь самый простой случай. Если допустить, что 
$$
\frac{\widehat \theta - \theta}{\sqrt{\mathsf D_\theta \widehat \theta}} \xrightarrow[n \to + \infty]{\mathsf d} \eta \sim \mathsf N_{0, 1},
$$
или близко к нормальному, а $\widehat{\mathsf{Var}}_B$ -- состоятельная оценка $\mathsf D_\theta \widehat \theta$, то асимптотический доверительный интервал уровня $(1 - \varepsilon)$ для параметра $\theta$ строится обычным образом и имеет вид
$$
\left(\widehat \theta - \tau_{1 - \varepsilon/2}\sqrt{\widehat{\mathsf{Var}}_B}, \ \widehat \theta + \tau_{1 - \varepsilon/2}\sqrt{\widehat{\mathsf{Var}}_B} \right),
$$
где $\tau_{1 - \varepsilon/2}$ -- квантиль уровня $(1 - \varepsilon/2)$ стандартного нормального распределения.


\subsubsection{Некоторые финальные замечания}
Рассмотрению методов ресемплинга можно посвятить не одну лекцию. Рассмотренные нами моменты и подходы -- это лишь часть тех бонусов, что дает исследователю бутстрэп. Аналогично тому, как было сделано в методе джекнайф, бутстрэп может оценивать и улучшать смещение исходной оценки $\widehat \theta$ и многое другое. 

Кроме того, все, что обсуждалось ранее, относится к непараметрическому бутстрэпу -- мы не выдвигаем никаких предположений о распределении $\xi$. В то же время, если известно, каким образом распределение $\xi$ зависит от интересующего нас параметра, метод может быть существенно усилен. Бутстрэп выборки в этом случае генерируются из эмпирической функции распределения, в которую вместо неизвестного параметра $\theta$ может быть подставлена его оценка. В этом случае количество бутстрэп выборок может быть сколь угодно большим.

Итак, теперь мы готовы посмотреть, как изученные методы применяются на практике к построению ансамблей моделей.
