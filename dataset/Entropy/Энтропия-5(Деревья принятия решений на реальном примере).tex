\documentclass[a4paper,14pt]{extarticle}


\usepackage[left = 20mm, right = 20mm, top = 25mm, bottom = 25mm, headheight=30pt]{geometry}
\usepackage{amssymb}
\usepackage{cmap}
\usepackage{enumerate}
\usepackage{latexsym}
\usepackage{amsmath}
\usepackage{euscript}
\usepackage{graphics}
\usepackage[T1, T2A]{fontenc}
\usepackage[utf8]{inputenc}
\usepackage[russian]{babel}
\usepackage[usenames]{color}
\usepackage{colortbl}
\usepackage{pgf,tikz}
\usepackage{multicol}
\usepackage{float}
\restylefloat{table}
\usepackage{multirow}
\usepackage{caption}
\usepackage{listings}
\usepackage{hyperref}
\usepackage{fancyhdr} 
\pagestyle{fancy}
\fancyhead{}
\fancyfoot{}
\fancyhead[LE,RO]{Университет ИТМО}
\fancyhead[RE,LO]{Высшая школа цифровой культуры}
\fancyfoot[LE,RO]{\thepage}
\renewcommand{\headrulewidth}{2pt}


\sloppy



\usepackage{xcolor} % Required for specifying custom colours
\definecolor{grey}{rgb}{0.9,0.9,0.9} % Colour of the box surrounding the title
%\usepackage[sfdefault]{ClearSans} % Use the Clear Sans font (sans serif)
%\usepackage{XCharter} % Use the XCharter font (serif)
\definecolor{foo}{HTML}{EFF5F9}

\DeclareMathOperator*{\argmax}{arg\,max}
\DeclareMathOperator*{\argmin}{arg\,min}
\DeclareMathOperator*{\Argmin}{Arg\,min}
\DeclareMathOperator*{\Argmax}{Arg\,max}
\DeclareMathOperator*{\logloss}{logloss}
\newcommand{\eqdef}{\overset{\mathrm{def}}{=\joinrel=}}
\def\re{{\rm Re}}
\def\im{{\rm Im}}
\def\dim{\rm dim}
\def\Ext{\rm Ext}
\def\wt#1{{{\widetilde #1} }}
\def\wh#1{{{\,\widehat #1\,} }}
\def\graph{{\rm gr\,}}
\def\ran{{\rm ran\,}}
\def\dom{{\rm dom\,}}
\def\ker{{\rm ker\,}}
\def\supp{{\rm supp\,}}
\def\diag{{\rm diag\,}}

\newcommand\dN{{\mathbb{N}}}
\newcommand\dR{{\mathbb{R}}}
\newcommand\dC{{\mathbb{C}}}
\newcommand{\bO}{{\mathbb{O}}}
\newcommand{\bU}{{\mathbb{U}}}
\newcommand\dZ{{\mathbb{Z}}}

\newcommand\gotB{{\mathfrak{B}}}
\newcommand\gotD{{\mathfrak{D}}}
\newcommand\gotH{{\mathfrak{H}}}
\newcommand\gotK{{\mathfrak{K}}}
\newcommand\gotL{{\mathfrak{L}}}
\newcommand\gotM{{\mathfrak{M}}}
\newcommand\gotN{{\mathfrak{N}}}
\newcommand\gotR{{\mathfrak{R}}}
\newcommand\gotS{{\mathfrak{S}}}
\newcommand\gotT{{\mathfrak{T}}}
\newcommand\gott{{\mathfrak{t}}}
\newcommand\gotC{{\mathfrak{C}}}
\newcommand\gotZ{{\mathfrak{Z}}}
\newcommand{\RNumb}[1]{\uppercase\expandafter{\romannumeral #1\relax}}

\newcommand{\ga}{{\alpha}}
\newcommand{\gd}{{\delta}}
\newcommand{\gD}{{\Delta}}
\newcommand{\gga}{{\gamma}}
\newcommand{\gG}{{\Gamma}}
\newcommand{\gF}{{\Phi}}
\newcommand{\gf}{{\phi}}
\newcommand{\gk}{{\kappa}}
\newcommand{\gK}{{\Kappa}}
\newcommand{\gl}{{\lambda}}
\newcommand{\gL}{{\Lambda}}
\newcommand{\gO}{{\Omega}}
\newcommand{\go}{{\omega}}
\newcommand{\gs}{{\sigma}}
\newcommand\gS{{\Sigma}}
\newcommand{\gT}{{\Theta}}
\newcommand{\gY}{{\Upsilon}}


\newcommand\cA{{\mathcal{A}}}
\newcommand\cB{{\mathcal{B}}}
\newcommand\cC{{\mathcal{C}}}
\newcommand\cD{{\mathcal{D}}}
\newcommand\cH{{\mathcal{H}}}
\newcommand\cK{{\mathcal{K}}}
\newcommand\cM{{\mathcal{M}}}
\newcommand\cN{{\mathcal{N}}}
\newcommand\cO{{\mathcal{O}}}
\newcommand\cP{{\mathcal{P}}}
\newcommand\cT{{\mathcal{T}}}
\newcommand\cU{{\mathcal{U}}}
\newcommand\cZ{{\mathcal{Z}}}

\newcommand\Rg{{\rm Rg}}
\newcommand\res{{\rm res}}
\newcommand\pr{{\rm Pr}}

\DeclareMathOperator{\sign}{sign}

\newtheorem{theorem}{Теорема}[subsection]
\newtheorem{lemma}{Лемма}[subsection]
\newtheorem{corollary}[theorem]{Следствие}
\newtheorem{definition}{Определение}[subsection]
\newtheorem{example}{Пример}[subsection]
\newtheorem{remark}{Замечание}[subsection]

\newcommand{\ba}{\begin{array}}
\newcommand{\ea}{\end{array}}

\newcommand{\bea}{\begin{eqnarray}}
\newcommand{\eea}{\end{eqnarray}}

\newcommand{\bead}{\begin{eqnarray*}}
\newcommand{\eead}{\end{eqnarray*}}

\newcommand{\be}{\begin{equation}}
\newcommand{\ee}{\end{equation}}

\newcommand{\bed}{\begin{displaymath}}
\newcommand{\eed}{\end{displaymath}}

\newcommand{\bl}{\begin{lemma}}
\newcommand{\el}{\end{lemma}}

\newcommand{\bt}{\begin{theorem}}
\newcommand{\et}{\end{theorem}}

\newcommand{\bc}{\begin{corollary}}
\newcommand{\ec}{\end{corollary}}

\newcommand{\br}{\begin{remark}}
\newcommand{\er}{\end{remark}}

\newcommand{\bd}{\begin{definition}}
\newcommand{\ed}{\end{definition}}

\newcommand{\bspi}{\begin{split}}
\newcommand{\espi}{\end{split}}

\newcommand{\la}{\label}
\newcommand{\rpm}{\raisebox{.2ex}{$\scriptstyle\pm$}}


\newenvironment{proof}%
{\begin{sloppypar}\noindent{\bf Доказательство.}}%
{\hspace*{\fill}$\square$\end{sloppypar}}
\renewcommand{\Large}{\fontsize{16}{25pt}\selectfont}

\newcommand{\slim}{\,\mbox{\rm s-}\hspace{-2pt} \lim}
\newcommand{\wlim}{\,\mbox{\rm w-}\hspace{-2pt} \lim}
\newcommand{\olim}{\,\mbox{\rm o-}\hspace{-2pt} \lim}
\newcommand{\transpose}[1]{\ensuremath{#1^{\scriptscriptstyle t}}}




\graphicspath{ {img/} }

\usepackage{indentfirst} 
\setlength\parindent{1cm}

\usepackage{textcase} 
\usepackage{titlesec}

\usepackage{indentfirst} 
\setlength\parindent{1cm}
% НАСТРОЙКИ ЗАГОЛОВКОВ
\usepackage{textcase} 
\usepackage{titlesec}
\titleformat{\section}[block]{\sffamily\Large\bfseries\filcenter}{\thesection}{0.5em}{}
\titleformat{\subsection}[block]{\large\sffamily\bfseries}{\thesubsection}{0.5em}{}
\titlespacing{\subsection}{2cm}{1mm}{3mm}

\begin{document}

%----------------------------------------------------------------------------------------
%	TITLE PAGE
%----------------------------------------------------------------------------------------

\begin{titlepage} % Suppresses displaying the page number on the title page and the subsequent page counts as page 1
	
	%------------------------------------------------
	%	Grey title box
	%------------------------------------------------
	\begin{center}
	\includegraphics[width=0.4\textwidth]{dc.png}
    \end{center}
    
    	\vfill
    	
	\colorbox{foo}{
		\parbox[t]{0.93\textwidth}{ % Outer full width box
			\parbox[t]{0.91\textwidth}{ % Inner box for inner right text margin
				\raggedleft % Right align the text
				%\fontsize{50pt}{80pt}\selectfont % Title font size, the first argument is the font size and the second is the line spacing, adjust depending on title length
				\vspace{0.7cm} % Space between the start of the title and the top of the grey box
				
				\huge Энтропия\\ Деревья принятия решений
				
				\vspace{0.7cm} % Space between the end of the title and the bottom of the grey box
			}
		}
	}
	
	\vfill % Space between the title box and author information
	
	%------------------------------------------------
	%	Author name and information
	%------------------------------------------------

    

	
	\parbox[t]{0.93\textwidth}{ % Box to inset this section slightly
		\raggedleft % Right align the text
		%\large % Increase the font size
		{Высшая Школа Цифровой Культуры}\\[4pt] % Extra space after name
		Университет ИТМО\\[4pt] % Extra space before URL
		dc@itmo.ru\\[4pt]
		
		\hfill\rule{0.2\linewidth}{1pt}% Horizontal line, first argument width, second thickness
	}
	
\end{titlepage}

%----------------------------------------------------------------------------------------


\newpage
%
\pagestyle{empty}
\tableofcontents
%
\clearpage
\pagestyle{fancy}


\section{Деревья принятия решений на реальном примере}
Одним из примеров применения деревьев принятия решений являются системы скоринга, используемые банками для оценки клиентов. Например, некий банк имеет клиентскую базу, и хочет понять, какие клиенты согласятся на оформление потребительского кредита в будущем, а какие нет. 

Рассмотрим данные (рисунок \ref{data}) о текущих клиентах банка\footnote{https://www.kaggle.com/itsmesunil/bank-loan-modelling} -- всего 12 предикторов, среди которых: возраст, опыт работы, доход, семейное положение, образование и проч., и отклик -- $Personal Loan$, принимающий значений 0, если клиент отказался оформить потребительский кредит, и 1 -- если согласился. Исходный объем данных содержит $5000$ объектов. Для дальнейшей оценки качества модели, мы его разделили на тренировочные и тестовые данные в отношении $70 \%$ на $30 \%$. Таким образом, обучение модели осуществляется на $3500$ объектах ($3158$ из которых относятся к классу $0$ -- клиенты отказавшиеся оформлять потребительский кредит, остальные к классу $1$ -- согласились). 
\begin{figure}[h]
\centering
\includegraphics[width=0.9\textwidth]{bank_data.png}
\caption{Пример тренировочных данных.}
\label{data}
\end{figure}
Отметим, что все предикторы принимают числовые значения, то есть являются некатегориальными. Будем строить бинарное дерево решений. В качестве критерия разделения будем использовать критерий $X_i \leq C$, где $C$ -- пороговый параметр из диапазона значений, принимаемых предиктором $X_i$. Так, на первом уровне, самым информативным является разделение по доходу (предиктор $Income$) с пороговым значением $92.5$ тысячи долларов в год. Такое разделение приводит к достаточно информативному множеству $S_{11}$ с близкой к нулю энтропией, второе же множество -- $S_{12}$ обладает высокой степенью неопределенности.
\begin{figure}[h]
\centering
\includegraphics[width=1\textwidth]{bank_tree_e.png}
\caption{Дерево принятия решений (энтропия).}
\label{bank_tree_e}
\end{figure}

Теперь построим второй уровень дерева. Для $S_{11}$ самым информативным будет разделение по предиктору $CCAvg$ -- расходы по кредитным картам в месяц. Для $S_{12}$ -- $Education$ -- уровень образования (он принимает значения $1$ -- для студентов, $2$ -- для выпускников и $3$ -- для имеющих профессиональное образование (степень)). В результате, уже на втором уровне мы получаем подмножество из $2393$ клиентов, которые отнесены к классу $0$ (левый блок с нулевой энтропией) -- это клиенты, которые не соглашались на оформление потребительского кредита, имели доход не более $92.5$ тысяч долларов в год и незначительные расходы по кредитной карте -- не более $2.95$ тысячи долларов в месяц.

Третий уровень дерева построен для оставшихся множеств с ненулевой энтропией -- множеств $S_{22}, S_{23}, S_{24}$. На этом уровне тоже получаем множество с нулевой энтропией (правый нижний блок на рисунке \ref{bank_tree_e}). На этот раз блок отвечает клиентам, согласившимся оформить потребительский кредит. Что это за клиенты, согласно построенному дереву? Это клиенты с высоким доходом, более $116.5$ тысяч долларов в год и либо выпускники, либо имеющие дополнительный уровень образования (степень).

В остальных подмножествах присутствуют разные клиенты, и мы можем либо продолжить построение дерева, либо закончить на трех уровнях. В таком случае класс назначается, как обычно, исходя из <<большинства>>.

На основании классификации тестовых данных, по построенному дереву может быть составлена матрица ошибок:
\begin{table}[h]
\centering
\begin{tabular}{|c|c|c|c|}
\hline
\multicolumn{2}{|c|}{\multirow{2}{*}{\textbf{Матрица ошибок}}} & \multicolumn{2}{c|}{Исходный класс} \\ \cline{3-4} 
\multicolumn{2}{|c|}{} & + & -- \\ \hline
\multirow{2}{*}{Прогноз} & + & TP=108 & FP=7 \\ \cline{2-4} 
 & -- & FN=30 & TN=1355 \\ \hline
\end{tabular}
\end{table}
Полученные результаты свидетельствуют о высокой точности модели
$$
Precision =\frac{TP}{TP + FP}=\frac{108}{108 + 7} \approx 0.9391,
$$
а также полноте
$$
Recall =\frac{TP}{TP + FN}=\frac{108}{108 + 30} \approx 0.7826.
$$

 
Обучение на аналогичных данных, но с использованием неопределенности Джини, приводит к результату, отображенному на рисунке \ref{bank_tree_g}. Можно заметить, что многие критерии схожи с ранее построенным деревом.
\begin{figure}[h]
\centering
\includegraphics[width=1\textwidth]{bank_tree_g.png}
\caption{Дерево принятия решений (Джини).}
\label{bank_tree_g}
\end{figure}
Кроме того, классификация тестовых данных дает тоже схожий результат:
\begin{table}[h]
\centering
\begin{tabular}{|c|c|c|c|}
\hline
\multicolumn{2}{|c|}{\multirow{2}{*}{\textbf{Матрица ошибок}}} & \multicolumn{2}{c|}{Исходный класс} \\ \cline{3-4} 
\multicolumn{2}{|c|}{} & + & -- \\ \hline
\multirow{2}{*}{Прогноз} & + & TP=113 & FP=4 \\ \cline{2-4} 
 & -- & FN=25 & TN=1358 \\ \hline
\end{tabular}
\end{table}
Точность и полнота немного, но повысились:
$$
Precision =\frac{TP}{TP + FP}=\frac{113}{113 + 4} \approx 0.9658,
$$
$$
Recall =\frac{TP}{TP + FN}=\frac{113}{113 + 25} \approx 0.8188.
$$

\section{Заключение}
Итак, в этой лекции мы осветили еще один подход к классификации -- деревья принятия решений, основная сфера применения которых -- поддержка процессов принятия управленческих решений. Метод не наделен такими тонкостями, как подбор гиперпараметров модели, и позволяет быстро обучить модель. Выбор той или иной меры неопределенности существенно не сказывается на результатах классификации, а ограничения по глубине дерева или количеству элементов в каждом узле дерева являются эвристическими подходами, т.е. не гарантируют лучшего результата или вообще работают только в каких-то частных случаях. Кроме того, обоснованных рекомендаций по тому, какой метод лучше работает, в настоящее время не существует, и окончательное решение остается на откуп исследователю.

\end{document}

